\documentclass[report]{•}
\begin{document}
\title{Fiche Récap' NITROX Elémentaire: \\
Utilisation d'un gaz enrichie en oxygène en plongée}

\section{Aspect Légale}
Le Nitrox est un gaz enrichie en oxygène. Le \% d'O$_2$ du mélange est alors supérieur à celui de l'air qui est de 20,9\%.

En France, La FFESSM prévoie 3 niveaux pour l'utilisation de ce type de mélange :
\begin{itemize}
\item Nitrox élémentaire 
\end{itemize}
Accessible à partir du N2, il permet l'utilisation d'un gaz dont le taux d'oxygène du mélange est compris entre 20.9 et 40\%. Un seul gaz pour toute la plongée la plongée. Pas d'équipement spécifique requis 
\begin{itemize}
\item Nitrox confirmé
\end{itemize}
Accessible à partir du NX élémentaire (+4 plongée dans cette configuration) il permet l'utilisation de tout type de mélanges binaires, avec plusieurs gaz par plongée (en générale 2). Nécessite un ordinateur, matériel compatible (blocs et détendeurs), outils de planifications optionnels.  

\begin{itemize}
\item Moniteur Nitrox
\end{itemize}
Accessible à partir du NX confimé + MF1. Permet l'enseignement, et la direction de plongée au nitrox. 

\textbf{Nous traiterons ici du premier niveau, c'est à dire du NITROX élementaire}

\section{Rappels et Lexique}

NITROX : NITRogen (azote en anglais) / OXygen

\textit{f}O$_2$ :fraction d'oxygène du mélange
\textit{f}N$_2$ :fraction d'azote du mélange

NX_{y} (ou EAN_{y} pour les anglo-saxon) désigne un gaz dont la \textit{f}O$_2$ est de y. 

EX: NX30 --> 30\% d'O_{2}

--------------------------------

\textit{PP}O$_2$ : pression partiel d'oxygène
\textit{PP}N$_2$ : pression partiel d'azote 

P$_Hydro$ = Prof. / 10 

P$_Abs$ = 1 + P$_Hydro$

Avec \textit{PP}gaz = \textit{f}gaz x P$_Abs$  (Lois de Dalton) 

EX: NX30 à 30 m --> \textit{PP}O$_2$= 0.3 x 4 = 1.2 bar   ///  \textit{PP}N$_2$ 0.7 x 4 = 2.8 bar 

--------------------------------

/!\/!\/!\ \textit{PP}O$_2$ > 1.6 bar est TOXIQUE /!\/!\/!\

On considère alors une\textit{PP}O$_2$FOND max de 1.5b, 1.4 conseillé 

PMU (profondeur maximal d'utilisation d'un gaz) (=MOD Maximum operating depth) 

PMU =  \textit{PP}O$_2$ / \textit{f}O$_2$ 

EX: NX30 à 1.6b -->  PMU = 1.6 / 0.3 = 5.3 soit 43.3 m soit 43m

--------------------------------

BestMix (gaz optimale à une profondeur donnée)  = \textit{PP}O$_2$ visée / P$_Abs$ 

EX 1.5 bar à 40 m --> 1.5 / 5 = 0.3 soit NX30

--------------------------------

PEA : Profondeur Equivalente à l'Air : effet narcotique du gaz utilisée // calcul de la décompression réel

PEA = ((P$_Abs$ x \textit{f}N$_2$ / 0.791) x 10) -10

EX 1 : à l'air 
\textit{PP}N$_2$ = 1- 0.209 = 0.791
à 50 m  P$_Abs$ = 5+1 =6 bar
PEA = ((6 x 0.791 / 0.791) x 10)-10 = 50

EX 2 : NX26 
\textit{PP}N$_2$ = 1- 0.26 = 0.74
à 50 m  P$_Abs$ = 5+1 =6 bar
PEA = ((6 x 0.74 / 0.791) x10)-10 = 46.1m soit une entrée dans les tables à 47m 

\end{document}